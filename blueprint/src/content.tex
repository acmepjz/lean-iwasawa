% In this file you should put the actual content of the blueprint.
% It will be used both by the web and the print version.
% It should *not* include the \begin{document}
%
% If you want to split the blueprint content into several files then
% the current file can be a simple sequence of \input. Otherwise It
% can start with a \section or \chapter for instance.

\tableofcontents

\section{Introduction}

\subsection{Iwasawa's theorem on growth of class groups}

The first goal of this project is formalize the proof of
Iwasawa's theorem on growth of class groups in a $\BZ_p$-extension tower
in {\Lean}.

\begin{definition}
\label{Zp-ext-def}
\lean{IsZpExtension,IsMvZpExtension.Kn}
\leanok
Let $K$ be a field, $p$ be a prime.

{\rm(i)} An extension $K_\infty/K$ is called a $\BZ_p$-extension,
if it is Galois with $\Gamma=\Gal(K_\infty/K)\cong\BZ_p$
as topological groups.

{\rm(ii)} If $K_\infty/K$ is a $\BZ_p$-extension, then for any
$n\geq 0$, define $K_n:=K_\infty^{\Gamma^{p^n}}$.
\end{definition}

\begin{thm}[Iwasawa's theorem]
\label{clgp-growth}
\uses{Zp-ext-def,p:f-g-torsion2,coinvariant-growth,X-inf-coinv-is-X-n}
\lean{IsZpExtension.multiplicity_classNumber_Kn_eq}
\leanok
Let $K$ be a number field, $p$ be a prime,
$K_\infty/K$ be a $\BZ_p$-extension.
Then there exist integers $\lambda,\mu,\nu$
such that for all sufficiently large $n$,
$\ord_p(\#\Cl(K_n))=\mu p^n+\lambda n+\nu$.
\end{thm}

\begin{proof}
This comes from Theorem \ref{p:f-g-torsion2},
Proposition \ref{coinvariant-growth},
and Proposition \ref{X-inf-coinv-is-X-n}.
\end{proof}

\section{Preliminaries in commutative algebra}

\subsection{Random}

\begin{prop}
\label{noeth-ring-filtration}
\lean{IsNoetherianRing.exists_relSeries_isQuotientEquivQuotientPrime}
\leanok
Let $A$ be a Noetherian ring and $M$ be a finitely generated $A$-module.
Then there exists a chain of submodules of $M$
\[0 = M_0 \subset M_1 \subset \cdots \subset M_n = M\]
such that for each $1\leq i\leq n$, $M_i/M_{i-1} \cong A/\fp_i$ for some prime ideal $\fp_i\in\Spec(A)$.
\end{prop}

\begin{proof}
\leanok
When $M=0$ we take $n=0$ and there is nothing to prove.
When $M\neq 0$ we have $\Ass(M)\neq\varnothing$
and we can take $\fp_1$ be any element in $\Ass(M)$
and we obtain $M_1:=A/\fp_1\hookrightarrow M$.
If $M_1\neq M$, do the same for $M/M_1$ to get $M_2/M_1\cong A/\fp_2\hookrightarrow M/M_1$.
Since $M$ is a Noetherian $A$-module,
the chain $M_1\subset M_2\subset\cdots$ must stop after a finite number of steps.
\end{proof}

\begin{thm}[Krull's Hauptidealsatz]
\label{krull-principal-ideal-thm}
Let $A$ be a Noetherian ring.
If $\fa=(a_1,\cdots,a_n)$ is an ideal of $A$ generated by $n$ elements,
and $\fp$ is a minimal prime over-ideal of $\fa$, then $\height(\fp)\leq n$.

Conversely, if $\fp$ is a prime ideal of $A$ of height $\leq n$,
then there exists an ideal $\fa$ of $A$ generated by $n$ elements, such that
$\overline\fp$ is a minimal prime in $A/\fa$.

(This is WIP in \PR{23778}.)
\end{thm}

\begin{proof}
If $\fp$ is a minimal prime over-ideal of $\fa$, then $\fa A_\fp\subset\fp A_\fp$,
and $\fp(A_\fp/\fa A_\fp)$, which is the only maximal ideal of
$A_\fp/\fa A_\fp$, is also a minimal prime ideal.
Therefore $A_\fp/\fa A_\fp$ is Artinian, and so
$\fa A_\fp$ is an open ideal. It follows that $\height(\fp)=\dim(A_\fp)\leq n$.

Conversely, if $\fp$ is a prime ideal of height $\leq n$,
then $\dim(A_\fp)\leq n$, so there exists an \emph{ideal of definition}
$IA_\fp$ of $A_\fp$
(i.e.~an open ideal with respect to $\fp A_\fp$-adic topology),
generated by $n$ elements.
Clearing their denominators we may assume that the generators take the form
$\frac{x_1}s,\cdots,\frac{x_n}s$ for some $x_1,\cdots,x_n\in I$
and $s\in A\setminus\fp$.
Since $\fp^NA_\fp\subset IA_\fp\subset\fp A_\fp$ for some integer $N\geq 1$,
we obtain $\fp^N\subset I\subset\fp$, and
$IA_\fp=(x_1,\cdots,x_n)A_\fp$
(hence we may replace $I$ by $(x_1,\cdots,x_n)$), and that $\fp$ is a minimal prime over-ideal of $I$
(since if $I\subset\fp'\subset\fp$ for some prime ideal $\fp'$,
then $\fp^N\subset\fp'$, hence $\fp\subset\fp'$, so $\fp=\fp'$).
\end{proof}

\begin{thm}
\label{UFD-iff-ht-1-principal}
\uses{krull-principal-ideal-thm}
Let $A$ be a Noetherian domain. Then $A$ is a UFD if and only if every prime ideal of height $1$ is principal
(recall that a prime ideal $\fp$ is of height $1$ if $\fp \neq 0$ and there is no prime ideal lies in between $0$ and $\fp$).

(We only need ``$\Rightarrow$'' in our project.)
\end{thm}

\begin{proof}
``$\Rightarrow$'':
Let $\fp$ be a prime ideal of height $1$,
and let $a\in\fp$ be a non-zero element.
Factor $a$ into a product of irreducible elements
(= prime elements) $a=p_1\cdots p_r$.
Then by $a\in\fp$ we obtain $p_i\in\fp$ for some $i$,
since $(\fp_i)\neq 0$ is a prime ideal contained in $\fp$, we have $\fp=(p_i)$.

``$\Leftarrow$'':
It suffices to show that every irreducible element is prime.
Let $p\in A$ be irreducible, and $\fp$ be a prime ideal of $A$ minimal containing $p$.
Then $\fp$ is of height $1$ by Krull's Hauptidealsatz
(Theorem \ref{krull-principal-ideal-thm}),
so $\fp=(\pi)$ is principal.
Since $p\in\fp=(\pi)$, $\pi\mid p$, and since $p$ is irreducible,
$p=u\pi$ for some unit $u$, therefore $p$ is prime (which generates $\fp$).
\end{proof}

\begin{thm}
\label{regular-local-ring-is-UFD}
A regular local ring is a UFD.
\end{thm}

\begin{lem}[Nakayama lemma, pro-$p$ version]\label{p:nak}
Suppose $\Lambda\cong\BZ_p[[T]]$,
$X$ is a pro-$p$ $\Lambda$-module,
$x_1,\cdots,x_t\in X$ such that their images in $X/TX$
generate $X/TX$ as a $\Lambda/(T)$-module.
Then $x_1,\cdots,x_t$ generate $X$ as a $\Lambda$-module.
\end{lem}

Similar result holds when $T$ is replaced
by a topologically nilpotent element.
Also for several variable formal power series.

\begin{proof}
Let $Y:=\Lambda x_1+\cdots+\Lambda x_t$ be the $\Lambda$-submodule
of $X$ generated by $x_1,\cdots,x_t$.
Then by the image of $Y$ in $X/TX$ is $X/TX$,
we know that $Y+TX=X$. Note that $Y$ is compact,
so it is closed in $X$, so $Z:=X/Y$ is a pro-$p$ abelian group,
and the image of $TX$ in $Z$ is $TZ$.
On the other hand, by $Y+TX=X$, the image of $TX$ in $Z$
is also $Z$, so $TZ=Z$. Since $T$ is topologically nilpotent in $\Lambda$,
for any open subgroup $U$ of $Z$, there exists $N\geq 0$ such that
for any $n\geq N$, $T^nZ\subset U$. But by $TZ=Z$ we know that
$T^nZ=Z$ for any $n\geq 0$. Therefore $Z$ must be zero.
\end{proof}

\subsection{Noetherian integrally closed domain}

\begin{prop}
\label{noeth-local-domain-is-icd-iff}
\leanok
For a Noetherian local domain $A$ of dimension one, the following are equivalent.
\begin{itemize}
\item
$A$ is integrally closed.
\item
The maximal ideal of $A$ is principal.
\item
$A$ is a discrete valuation ring.
\item
$A$ is a regular local ring.
\end{itemize}
(Mathlib: \docs{IsDiscreteValuationRing.TFAE} and
\docs{tfae_of_isNoetherianRing_of_isLocalRing_of_isDomain}.)
\end{prop}

\begin{proof}
\leanok
Omitted.
\end{proof}

\begin{definition}
\label{krull-domain-defn}
\lean{IsKrullDomain}
\leanok
An integral domain $A$ is called a \emph{Krull domain}
if it satisfies the following properties:
\begin{itemize}
\item
$A_\fp$ is a discrete valuation ring for all height one primes $\fp$ of $A$,
\item
$A=\bigcap_{\fp\in\Spec(A),\height(\fp)=1}A_\fp$ inside $\Frac(A)$,
\item
any nonzero element of $A$ is contained in only a finitely many height one primes of $A$.
\end{itemize}
\end{definition}

% check this IsDedekindDomain.HeightOneSpectrum.iInf_localization_eq_bot

\begin{lem}
\label{integrally-closed-is-local-prop}
Let $A$ be a domain, $S$ be a subset of $\Spec(A)$
such that $A_\fp$ is integrally closed for all $\fp\in S$
and $\bigcap_{\fp\in S}A_\fp=A$ inside $\Frac(A)$.
Then $A$ itself is integrally closed.

(Generalization of \docs{IsIntegrallyClosed.of_localization_maximal}. \PR{24588})
\end{lem}

\noindent\verb|#check |\docs{isIntegrallyClosed_of_isLocalization}

\noindent\verb|#check |\docs{PrimeSpectrum.localization_comap_injective}

\noindent\verb|#check |\docs{PrimeSpectrum.localization_comap_range}

\begin{proof}
Suppose $x\in\Frac(A)$ is integral over $A$. Then it's also integral over $A_\fp$
for all $\fp\in\Spec(A)$. Hence $x\in A_\fp$ for all $\fp\in S$.
So $x\in\bigcap_{\fp\in S}A_\fp=A$.
\end{proof}

\begin{prop}
\label{noeth-ring-is-krull-domain-iff}
\lean{isKrullDomain_iff_isIntegrallyClosed}
\leanok
\uses{noeth-local-domain-is-icd-iff,krull-domain-defn,integrally-closed-is-local-prop}
A Noetherian ring is a Krull domain if and only if it is an integrally closed domain.
\end{prop}

\begin{proof}
``$\Rightarrow$'': Lemma \ref{integrally-closed-is-local-prop}.

``$\Leftarrow$'': Let $\fp$ be a height one prime of $A$.
Then $A_\fp$ is integrally closed (\docs{isIntegrallyClosed_of_isLocalization}).
We have $\Spec(A_\fp)=\{0,\fp A_\fp\}$ hence
$A_\fp$ is also a Noetherian local domain of dimension one.
Now by Proposition \ref{noeth-local-domain-is-icd-iff}, $A_\fp$ is a DVR.

??? ???
\end{proof}

\subsection{Semilocal PID}

\begin{lem}
\label{semilocal-fg-mod}
\lean{Module.finite_of_finite_localized_maximal}
\leanok
Let $A$ be a semilocal ring, $M$ be an $A$-module such that
for any maximal ideal $\fm$ of $A$, $M_\fm$ is a finitely generated $A_\fm$-module.
Then $M$ is a finitely generated $A$-module.
\end{lem}

\begin{proof}
\leanok
Suppose $M_{\fm_i}$ is generated by $\{x_{i,1},\cdots, x_{i,n_i}\}$ for each maximal ideal $\fm_i$ of $A$.
Let $y_{i,k}$ be a numerator of $x_{i,k}$, which is in $A$.

We prove that $\bigcup_{i}\{y_{i,1} \cdots, y_{i,n_i}\}$
is a finite set (Since $A$ is semilocal) of generators of $M$.
Let $N$ be the $A$-submodule of $M$ generated by $\bigcup_{i}\{y_{i,1} \cdots, y_{i,n_i}\}$.

By local property (\docs{Submodule.eq_top_of_localization_maximal}), it suffices to show that
$N_{\fm_{i}} = M_{\fm_{i}}$ for all maximal ideals $\fm_i$ of $A$.
It's clear that $N_{\fm_{i}} \subseteq  M_{\fm_{i}}$, so we only need to show that $M_{\fm_{i}} \subseteq  N_{\fm_{i}}$.

In other words, need to prove that $M_{\fm_{i}} \subseteq
\operatorname{span}_{A_{\fm_i}} f_i (\bigcup_{i}\{y_{i,1} \cdots, y_{i,n_i}\})$
where $f_i : M \to M_{\fm_i}$ is the localization map.
It is enough to show the set of generators of $M_{\fm_{i}}$ is contained in right hand side.

Take any $x$ in the set of generators of $M_{\fm_{i}}$.
It remains to show that the image of a numerator of $x$ under $f_i$
is in the generators of right hand side by
\docs{Submodule.mem_of_numerator_image_mem} and \docs{Submodule.mem_span}.
i.e. $f_i(y) \in f_i (\bigcup_{i}\{y_{i,1} \cdots, y_{i,n_i}\})$
for some numerator $y$ of $x$.

We take $y$ as some $y_{i,k}$ as above.
By definition of $y_{i,k}$, we have $f_i(y_{i,k})$ in right hand side.
\end{proof}

\begin{lem}
\label{semilocal-PID}
\lean{isPrincipalIdealRing_of_isPrincipalIdealRing_localization}
\uses{semilocal-fg-mod}
\leanok
Let $A$ be a semilocal (i.e.~only finitely many maximal ideals)
integral domain,
such that for every maximal ideal $\fp$ of $A$, $A_\fp$ is a PID.
Then $A$ itself is a PID.
\end{lem}

\begin{proof}
\leanok
It's known that a semilocal Dedeking domain is a PID
(\docs{IsPrincipalIdealRing.of_finite_primes}).
So we only need to show $A$ is a Dedekind domain.

Let $I$ be any ideal of $A$. Apply Lemma \ref{semilocal-fg-mod} to $I$ we know that
$I$ is finitely generated. Hence $A$ is a Noetherian ring.

Let $\fp\neq 0$ be a prime ideal of $A$.
Choose a maximal ideal $\fm$ containing $\fp$.
Then $0\neq\fp A_\fm\subset\fm A_\fm$. Since $A_\fm$ is a PID
(hence DVR), we have
$\Spec(A_\fm)=\{0,\fm A_\fm\}$,
so know that $\fp A_\fm=\fm A_\fm$, hence $\fp=\fm$ is maximal.

It's known that if the localizations of a domain at all maximal ideals
are integrally closed, then the domain itself is integrally closed
(\docs{IsIntegrallyClosed.of_localization_maximal}).
Hence our $A$ is integrally closed. This completes the proof.
\end{proof}

\iffalse
\begin{proof}
Hint: for a maximal ideal
$\fp$ of $A$, $A_\fp$ is a PID, hence by Krull's intersection theorem,
$\bigcap_{n\geq 0}\fp^nA_\fp=0$,
and in particular, $\fp\supsetneqq\fp^2$. By Chinese remainder theorem,
for each $\fp$ we can choose $\pi_\fp\in A$ such that
$\pi_\fp\in\fp\setminus\fp^2$ and $\pi_\fp\equiv 1\pmod{\fq}$
for all maximal ideals $\fq\neq\fp$.
Prove that $\pi_\fp$ is a generator of $\fp A_\fp$ as an ideal of $A_\fp$.
Similar to Homework 3.3,
prove that if $\fa$ is an ideal of $A$, then
$\fa=\bigcap_\fp\fa A_\fp\subset\Frac(A)$ where $\fp$ runs over all maximal ideals
of $A$.
Now if $\fa\neq 0$, then for each maximal ideal $\fp$ of $A$,
there exists $f_\fp\in A_\fp\setminus\{0\}$ such that $\fa A_\fp=f_\fp A_\fp$,
and we may choose $f_\fp$ such that $f_\fp\in A\setminus\{0\}$.
Let $e_\fp\geq 0$
be the integer such that $f_\fp\in\fp^{e_\fp}\setminus\fp^{e_\fp+1}$.
Let $f=\prod_\fp\pi_\fp^{e_\fp}$ where $\fp$ runs over all maximal ideals
of $A$.
Prove that $fA_\fp=f_\fp A_\fp$ for all maximal ideals $\fp$ of $A$,
which enables us to deduce that $(f)=\bigcap_\fp fA_\fp=\bigcap_\fp\fa A_\fp
=\fa$.

Missing details in the hint:
\begin{itemize}
\item
Firstly for any $n\geq 1$, we have $\fp^nA_\fp\cap A=\fp^n$: the ``$\supset$''
is clear, and for ``$\subset$'', say $a\in A$ such that
$a=\frac{r}{s}\in\fp^nA_\fp\subset\Frac(A)$ for some $r\in\fp^n$ and $s\in A\setminus\fp$,
then we have $sa=r\equiv 0\pmod{\fp^n}$;
on the other hand, the prime ideals of the ring $A/\fp^n$ are one-to-one correspondence to the prime ideals $\fq$ of $A$ satisfying $\fp^n\subset\fq$,
which implies that $\fp=\sqrt{\fp^n}\subset\sqrt\fq=\fq$, since $\fp$ is a maximal ideal,
we must have $\fq=\fp$, namely $A/\fp^n$ is a local ring with maximal ideal
$\fp/\fp^n$, therefore $s$ is invertible in $A/\fp^n$ and which implies
$a\equiv 0\pmod{\fp^n}$.

Since $A_\fp$ is PID, $\fp A_\fp$ is principal, say $\fp A_\fp=gA_\fp$
for some $g\in\fp A_\fp$.
Then $\pi_\fp=gh$ for some $h\in A_\fp$.
We must have $h\notin\fp A_\fp$, otherwise $\pi_p\in\fp^2 A_\fp\cap A=\fp^2$, contradiction. Therefore $h\in A_\fp^\times$, which means that $\pi_\fp$ is also a generator of $\fp A_\fp$.

\item
The proof of $\fa=\bigcap_\fp\fa A_\fp$ is exactly the same as
Homework 3.3, which is omitted.

\item
Since $\pi_\fp$ is a generator of $\fp A_\fp$
and $\pi_\fq\in A_\fp^\times$ for all $\fq\neq\fp$,
we have $fA_\fp=\fp^{e_\fp}A_\fp$.
On the other hand, $f_\fp\in\fp^{e_\fp}\setminus\fp^{e_\fp+1}$ implies that
$f_\fp=\pi_\fp^{e_\fp}g_\fp$ for some $g_\fp\in A_\fp$
and we must have $g_\fp\notin\fp A_\fp$, otherwise $f_\fp\in\fp^{e_\fp+1}A_\fp\cap A=\fp^{e_\fp+1}$, contradiction. Therefore $g_\fp\in A_\fp^\times$,
hence $fA_\fp=\fp^{e_\fp}A_\fp=f_\fp A_\fp$.
\end{itemize}

\textbf{Homework 3.3.}
Let $A$ be an integral domain, $\Frac(A)$ be its fraction field.
Then for every maximal ideal $\fp$ of $A$,
the localization $A_\fp$ can be viewed as a subring of $\Frac(A)$.
Prove that $A=\bigcap_\fp A_\fp\subset\Frac(A)$
where $\fp$ runs over all maximal ideals of $A$.

\emph{Proof.}
Clearly ``$\subset$'' holds.
Conversely, suppose there exists $x\in\bigcap_\fp A_\fp\subset\Frac(A)$
but $x\notin A$,
consider $\fa=(A:x):=\{b\in A\mid bx\in A\}$,
then it is a proper ideal of $A$,
hence there exists a maximal ideal $\fm$ of $A$ containing $\fa$.
Write $x=\frac as\in A_\fm\subset\Frac(A)$
for some $a\in A$ and $s\in A\setminus\fm$,
then $sx=a\in A$, hence $s\in\fa\subset\fm$, a contradiction.
Therefore ``$\supset$'' also holds.

(\emph{Another proof.}
If $x\in\bigcap_\fp A_\fp\subset\Frac(A)$,
then for each maximal ideal $\fp$ of $A$,
we may write $x=\frac{r_\fp}{s_\fp}\in\Frac(A)$ for some $r_\fp\in A$
and $s_\fp\in A\setminus\fp$. Let $\fa$ be the ideal of $A$ generated by all the $s_\fp$'s.
Then $\fa$ is not contained in any maximal ideal $\fp$ of $\fa$,
hence we must have $\fa=(1)$, say $1=\sum_{i=1}^ns_it_i$
for a finite subset $\{\fp_1,\cdots,\fp_n\}$ of maximal ideals of $A$
and some $t_i\in A$, here for the simplicity of notation, denote
$s_i:=s_{\fp_i}$ and $r_i:=r_{\fp_i}$.
Therefore $x=\sum_{i=1}^nxs_it_i=\sum_{i=1}^nr_it_i\in A$.)
\end{proof}
\fi

\section{Structure of module up to pseudo-isomorphism}

\subsection{Characteristic ideal}

\begin{prop}
\label{char-ideal-preliminary}
\uses{noeth-ring-filtration}
\lean{Module.IsTorsion.finite_primeHeight_one_support,
Module.IsTorsion.isFiniteLength_localizedModule_of_primeHeight_le_one}
\leanok
Let $A$ be a Noetherian ring, $M$ be a
finitely generated torsion $A$-module.
Then for any height one prime $\fp$ of $A$,
$M_\fp$ is an $A_\fp$-module of finite length. Moreover, there are only finitely
many height one primes $\fp$ of $A$ such that $M_\fp\neq 0$.
\end{prop}

\begin{proof}
\leanok
By Proposition \ref{noeth-ring-filtration},
we may let $0=M_0\subset M_1\subset\cdots\subset M_n=M$ be a filtration of $M$
such that for each $1\leq i\leq n$,
$M_i/M_{i-1}\cong A/\fp_i$ for some prime ideal $\fp_i$ of $A$.
Note that if $\fp,\fq$ are prime ideals of $A$,
then $(A/\fp)_\fq\neq 0$ if and only if $\fp\subset\fq$.
Therefore by $M$ is torsion $A$-module, we obtain that $\height(\fp_i)\geq 1$
for all $1\leq i\leq n$,
and if $\fp$ is a height one prime, then $M_\fp\neq 0$
if and only if $\fp_i\subset\fp$ for some $i$,
by height considerations this means that $\fp_i=\fp$ for some $i$,
hence such $\fp$ are only finitely many.

To prove $\ell_{A_\fp}(M_\fp)<\infty$,
we only need to show that if $\fp,\fq$ are prime ideals of $A$
such that $\height(\fp)\geq 1$ and $\height(\fq)=1$,
then $(A/\fp)_\fq$ is an $A_\fq$-module of finite length.
In fact, by height considerations we know that $(A/\fp)_\fq\neq 0$
if and only if $\fp=\fq$, in this case $(A/\fq)_\fq=A_\fq/\fq A_\fq
=k(\fq)$ is the residue field of $\fq$, which is
an $A_\fq$-module of length one.

(\emph{Another proof without using Proposition \ref{noeth-ring-filtration}.}
Note that $M_\fp=0$ for all minimal prime ideals of $A$,
therefore if $\fp$ is of height one such that $M_\fp\neq 0$,
then $\fp$ is a minimal element in $\Supp(M)$, hence $\fp\in\Ass(M)$
which is a finite set.
So there are only finitely
many height one primes $\fp$ of $A$ such that $M_\fp\neq 0$.

Suppose $\fp$ is a height one prime such that $M_\fp\neq 0$.
To prove that $M_\fp$ is an $A_\fp$-module of finite length,
we only need to prove that the ring $A_\fp/\Ann_{A_\fp}(M_\fp)$
is Artinian. Note that $\Ann_{A_\fp}(M_\fp)=\Ann_A(M)_\fp$,
hence $A_\fp/\Ann_{A_\fp}(M_\fp)=(A/\Ann_A(M))_\fp$
whose prime ideals are one-to-one correspondence to prime ideals
of $A$ between $\Ann_A(M)$ and $\fp$,
i.e.~prime ideals in $\Supp(M)$ which is contained in $\fp$.
Such ideal can only be $\fp$ itself,
since $M$ is torsion, every prime ideal in $\Supp(M)$ has height $\geq 1$.
Hence $A_\fp/\Ann_{A_\fp}(M_\fp)$
is Artinian.)
\end{proof}

In particular, this allows us to define the \emph{characteristic ideal} of $M$.

\begin{definition}
\label{char-ideal}
\uses{char-ideal-preliminary}
\lean{Module.charIdeal}
\leanok
Let $A$ be a Noetherian ring, $M$ be a
finitely generated torsion $A$-module.
The \emph{characteristic ideal} of $M$,
denoted by $\chaR_A(M)$, or simply $\chaR(M)$ if there is no risk of confusion, is defined to be
$$
\chaR_A(M):=\prod_{\substack{\fp\in\Spec(A)\\
\height(\fp)=1}}\fp^{\ell_{A_\fp}(M_\fp)}.
$$
\end{definition}

\begin{prop}
\label{char-ideal-additive}
\uses{char-ideal}
\lean{Module.IsTorsion.injective,Module.IsTorsion.surjective,Module.IsTorsion.exact,
Module.IsTorsion.charIdeal_eq_mul_of_exact}
\leanok
Let $A$ be a Noetherian ring.
Let $0\to M'\to M\to M''\to 0$ be a short exact sequence of finitely generated $A$-modules. Then $M$ is $A$-torsion if and only if $M'$ and $M''$ are $A$-torsion.
If $M$ is $A$-torsion, then $\chaR_A(M)=\chaR_A(M')\chaR_A(M'')$.
\end{prop}

\begin{proof}
\leanok
Since localization is exact, for any prime ideal $\fp$ of $A$,
the $0\to M_\fp'\to M_\fp\to M_\fp''\to 0$ is exact.
Let $\fp$ runs over all minimal prime ideals of $A$,
we obtain that $M$ is $A$-torsion if and only if $M'$ and $M''$ are $A$-torsion.
Also, we have $\ell_{A_\fp}(M_\fp)=\ell_{A_\fp}(M_\fp')+\ell_{A_\fp}(M_\fp'')$,
hence $\chaR_A(M)=\chaR_A(M')\chaR_A(M'')$ holds.
\end{proof}

\subsection{Pseudo-null module}

\begin{definition}
\label{pseudo-null}
\lean{Module.IsPseudoNull, LinearMap.IsPseudoIsomorphism, Module.IsPseudoIsomorphic}
\leanok
Let $A$ be a Noetherian ring.

{\rm(i)}
A finitely generated $A$-module $M$ is called a \emph{pseudo-null} $A$-module,
if $M_\fp=0$ for all prime ideals $\fp$ of $A$ of height $\leq 1$.

{\rm(ii)}
An $A$-linear homomorphism $f:M\to N$ between finitely generated $A$-modules
is called a \emph{pseudo-isomorphism} (\emph{pis} for short),
if its kernel and cokernel are pseudo-null $A$-modules.

{\rm(iii)}
Two finitely generated $A$-modules $M,N$ are called
\emph{pseudo-isomorphic} (\emph{pis} for short),
denoted by $M\sim_\pis N$ or simply $M\sim N$,
if there exists a pseudo-isomorphism from $M$ to $N$.
\end{definition}

\begin{remark}
We warn the reader that $M\sim N$ not necessarily implies $N\sim M$.
\end{remark}

\begin{prop}
\label{pseudo-null-criterion}
\uses{pseudo-null}
\lean{Module.isPseudoNull_iff_subsingleton_of_krullDimLE_one,
Module.isPseudoNull_iff_finite_of_ringKrullDim_eq_two}
\leanok
Let $A$ be a Noetherian ring, $M$ be a finitely generated $A$-module.

{\rm(i)} If $A$ is of Krull dimension $\leq 1$, then $M$ is pseudo-null
if and only if $M=0$.

{\rm(ii)} If $A$ is of Krull dimension $2$,
is a local ring with finite residue field, then $M$ is pseudo-null
if and only if the cardinality of $M$ is finite.
\end{prop}

\begin{proof}
\leanok
(i) Clear.

(ii) Let $\fm$ be the maximal ideal of $A$.
If $M$ is finite, then there exists $r\in\BN$
such that $\fm^rM=0$, hence $\supp(M)\subset\{\fm\}$.
On the other hand, if $\supp(M)\subset\{\fm\}$,
then there exists $r\in\BN$
such that $\fm^rM=0$, hence $\fm^r\subset\Ann_A(M)$, therefore
$M$ is a finitely generated $A/\fm^r$-module, which must be finite.
\end{proof}

\begin{prop}
\label{pseudo-null-char-ideal}
\uses{pseudo-null,char-ideal-additive}
\lean{Module.charIdeal_eq_one_of_isPseudoNull,
LinearMap.IsPseudoIsomorphism.charIdeal_eq,
Module.IsPseudoIsomorphic.charIdeal_eq}
\leanok
Let $A$ be a Noetherian ring, $M$, $N$ be finitely generated torsion $A$-modules.

{\rm(i)} If $M$ is pseudo-null, then $\chaR_A(M)=0$.

{\rm(ii)} If $M\sim N$, then $\chaR_A(M)=\chaR_A(N)$.
\end{prop}

\begin{proof}
\leanok
Clear from definition and Proposition \ref{char-ideal-additive}.
\end{proof}

\subsection{Structure theorem}

\begin{definition}
\label{ht-1-localization-is-PID}
\lean{HeightOneLocalizationIsPID}
\leanok
A Noetherian ring $A$ is called
``height one localizations are PID'', if
for any finitely many height one primes $\fp_1,\cdots,\fp_r$ of $A$,
let $S:=A\setminus\bigcup_{i=1}^r\fp_i$, then $S^{-1}A$ is a PID.
\end{definition}

If $A$ is a Noetherian integral domain, then by Lemma \ref{pis-iff} we
only need to check $r=1$ case.

\begin{lem}
\label{pis-iff}
\uses{char-ideal-preliminary,pseudo-null}
\lean{LinearMap.isPseudoIsomorphism_iff_bijective_map}
\leanok
Let $A$ be a Noetherian ring and let
$M,N$ be finitely generated torsion $A$-modules.
Let $\Sigma=\{\fq_1,\cdots,\fq_r\}
=\{\fq\in\Supp(M)\cup\Supp(N)\mid\height(\fq)=1\}$
(by Proposition \ref{char-ideal-preliminary} this is a finite set).
Let $S:=A\setminus\bigcup_{i=1}^r\fq_i$ which is a multiplicative subset of $A$.
Let $f:M\to N$ be an $A$-module homomorphism.
Then $f$ is a pseudo-isomorphism if and only if $S^{-1}f:S^{-1}M\to S^{-1}N$
is an isomorphism.
\end{lem}

\begin{proof}
Since the height one support of $\ker(f)$ and $\coker(f)$
are contained in $\Sigma$,
and since $S^{-1}\ker(f)=\ker(S^{-1}f)$,
$S^{-1}\coker(f)=\coker(S^{-1}f)$ (localization is exact),
we only need to prove that if $M$ is a finitely generated torsion $A$-module
whose height one support is contained in $\Sigma$,
then $S^{-1}M=0$ if and only if $M$ is pseudo-null (equivalently, $M_\fq=0$
for all $\fq\in\Sigma$):
``$\Rightarrow$'': Clear.
``$\Leftarrow$'': For all $\fq\in\Sigma$, $M_\fq=0$ means that $\Ann(M)\not\subset\fq$,
since $\fq$ are prime ideals, we have $\Ann(M)\not\subset\bigcup_{\fq\in\Sigma}\fq$,
so $S^{-1}M=0$.
\end{proof}

\begin{prop}[Structure theorem of finitely generated torsion $A$-modules]
\label{structure-thm}
\uses{ht-1-localization-is-PID,char-ideal,pseudo-null}
\lean{Module.IsTorsion.isPseudoIsomorphic_pi}
\leanok
Let $A$ be a Noetherian ring whose height one localizations are PID
(Definition \ref{ht-1-localization-is-PID}).
Let $M$ be a finitely generated torsion $A$-module.
Then there exist height one primes $\fp_1,\cdots,\fp_s$ of $A$
and positive integers $k_1,\cdots,k_s$, such that there exists a pseudo-isomorphism
$M\to\bigoplus_{i=1}^s A/\fp_i^{k_i}$.
Moreover, the sequence $(\fp_i^{k_i})_{i=1}^s$ is unique up to ordering.
\end{prop}

\begin{proof}
Let
$\Sigma=\{\fq_1,\cdots,\fq_r\}=\{\fq\in\Supp(M)\mid\height(\fq)=1\}$
(by Proposition \ref{char-ideal-preliminary} this is a finite set),
and let $S=A\setminus\bigcup_{i=1}^r\fq_i$.
Then $S^{-1}M$ is a finitely generated $S^{-1}A$-module,
and is torsion (for example, since $\Hom_{S^{-1}A}(S^{-1}M,S^{-1}A)
=S^{-1}\Hom_A(M,A)=0$).

Note that the prime ideals $\fP$ of $S^{-1}A$ are one-to-one correspondence
to prime ideals $\fp$ of $A$ satisfying $\fp\cap S=\varnothing$
(i.e.~$\fp\subset\bigcup_{i=1}^r\fq_i$, i.e~$\fp\subset\fq_i$ for some $i$),
by $\fP=S^{-1}\fp$ and $\fp=\fP\cap A$.
In particular, $S^{-1}A$ is of dimension $\leq 1$.

By structure theorem of finitely generated torsion modules over a PID,
there exist primes $\fp_1,\cdots,\fp_s$ of $A$
satisfying $\fp_i\cap S=\varnothing$,
and positive integers $k_1,\cdots,k_s$, such that
there exists an isomorphism
$g:S^{-1}M\xrightarrow\sim\bigoplus_{i=1}^sS^{-1}(A/\fp_i^{k_i})$
of $S^{-1}A$-modules.
Since $S^{-1}M$ is torsion, the $\fp_i$'s must be of height one.
Since $\Hom_{S^{-1}A}(S^{-1}M,\bigoplus_{i=1}^sS^{-1}(A/\fp_i^{k_i}))
=S^{-1}\Hom_A(M,\bigoplus_{i=1}^sA/\fp_i^{k_i})$,
by multiplying an element of $S$ to $g$ if necessary
(this doesn't change the fact that $g$ is an isomorphism), we may find
an $A$-linear map $f:M\to\bigoplus_{i=1}^sA/\fp_i^{k_i}$
such that $g=S^{-1}f$.
Now by (i) we know that $f$ is a pseudo-isomorphism.

Conversely, if $(\fp_i^{k_i})_{i=1}^s$ is such that there exists a
pseudo-isomorphism
$M\to\bigoplus_{i=1}^s A/\fp_i^{k_i}$,
then enlarging $S$ if necessary, by Lemma \ref{pis-iff},
its localization
$S^{-1}M\to\bigoplus_{i=1}^sS^{-1}(A/\fp_i^{k_i})$
is an isomorphism of $S^{-1}A$-module, hence by
structure theorem of finitely generated torsion modules over a PID,
the $(\fp_i^{k_i})_{i=1}^s$ is unique up to ordering.
\end{proof}

\begin{prop}
\label{pis-symm}
\uses{ht-1-localization-is-PID,pis-iff}
\lean{Module.IsTorsion.isPseudoIsomorphic_iff_nonempty_linearEquiv_localizedModule,
Module.IsPseudoIsomorphic.symm,
Module.isPseudoIsomorphic_comm}
\leanok
Let $A$ be a Noetherian ring whose height one localizations are PID
(Definition \ref{ht-1-localization-is-PID}).
Let $M,N$ be finitely generated torsion $A$-modules.
Then the followings are equivalent:
\begin{itemize}
\item[(a)]
There exists a pseudo-isomorphism $M\to N$.
\item[(b)]
For any height one prime $\fp$ of $A$, we have $M_\fp\cong N_\fp$.
\end{itemize}
In particular, if there exists a pseudo-isomorphism $M\to N$,
then there also exists a pseudo-isomorphism $N\to M$.
\end{prop}

\begin{proof}
(a)$\Rightarrow$(b): Clear.

(b)$\Rightarrow$(a): Let $\Sigma=\{\fq_1,\cdots,\fq_r\}
=\{\fq\in\Supp(M)\cup\Supp(N)\mid\height(\fq)=1\}$
(by Proposition \ref{char-ideal-preliminary} this is a finite set),
and let $S=A\setminus\bigcup_{i=1}^r\fq_i$.
Since $M_\fp\cong N_\fp$ for all height one primes $\fp$ of $A$,
the $S^{-1}M$ and $S^{-1}N$, being finitely generated torsion modules over
a PID $S^{-1}A$, are isomorphic. Say $g:S^{-1}M\xrightarrow\sim S^{-1}N$
is an isomorphism of $S^{-1}A$-modules.
Since $\Hom_{S^{-1}A}(S^{-1}M,S^{-1}N)=S^{-1}\Hom_A(M,N)$,
by multiplying an element of $S$ to $g$ if necessary
(this doesn't change the fact that $g$ is an isomorphism), we may find
an $A$-linear map $f:M\to N$
such that $g=S^{-1}f$.
Now by Lemma \ref{pis-iff} we know that $f$ is a pseudo-isomorphism.
\end{proof}

\subsection{Noetherian regular domain}

\begin{prop}
\label{regular-domain-is-good}
\lean{IsKrullDomain.heightOneLocalizationIsPID}
\leanok
\uses{semilocal-PID,ht-1-localization-is-PID,noeth-ring-is-krull-domain-iff}
Let $A$ be a Noetherian regular domain
(more generally, a Noetherian integrally closed domain).
Then height one localizations of $A$ are PID
(Definition \ref{ht-1-localization-is-PID}).
\end{prop}

\begin{proof}
\leanok
If $A$ is a Noetherian regular domain
(more generally, a Noetherian integrally closed domain),
then for each height one prime $\fp$ of $A$,
$A_\fp$ is a PID.
For any finitely many height one primes $\fp_1,\cdots,\fp_r$ of $A$,
let $S:=A\setminus\bigcup_{i=1}^r\fp_i$,
then $S^{-1}A$ is a semilocal integral domain, whose
maximal ideals are $S^{-1}\fp_1,\cdots,S^{-1}\fp_r$,
and we have $(S^{-1}A)_{S^{-1}\fp_i}=A_{\fp_i}$,
therefore by Lemma \ref{semilocal-PID} we know that $S^{-1}A$ is a PID.
\end{proof}

\section{Structure of Iwasawa module}

\subsection{Iwasawa algebra}

Let $p$ be a prime,
$\Gamma$ be a topological abelian group, isomorphic to $\BZ_p$
as a topological abelian group, and let $\gamma\in\Gamma$ be a topological generator
of $\Gamma$ (i.e.~the subgroup $\{\gamma^n\mid n\in\BZ\}$ is dense in $\Gamma$).
For each $n\geq 0$ let $\Gamma_n:=\Gamma/\Gamma^{p^n}$.

\begin{definition}
\label{Iwasawa-alg-defn}
The \emph{Iwasawa algebra} is defined as the completed group algebra
$$
\Lambda:=\BZ_p[[\Gamma]]:=\varprojlim_n\BZ_p[\Gamma_n],
$$
where the transition map $\BZ_p[\Gamma_{n+1}]\to
\BZ_p[\Gamma_n]$ is induced by the natural projection
$\Gamma_{n+1}\to\Gamma_n$.
Each $\BZ_p[\Gamma_n]$ is a free $\BZ_p$-module of rank $p^n$,
we endow it with the $p$-adic topology,
and endow $\Lambda$ with the subspace topology
of the product topology of $\prod_{n=0}^\infty\BZ_p[\Gamma_n]$.
\end{definition}

\begin{prop}
\label{linear-map-is-adic-continuous}
Let $A$ be a ring, $\fa$ be an ideal of $A$.
Let $M,N$ be two $A$-modules,
and $\varphi:M\to N$ be an $A$-module homomorphism.
Then $\varphi$ is continuous if we endow $M,N$ with $\fa$-adic topology.

In particular, if $\varphi$ is an $A$-module isomorphism,
then it is a homeomorphism of topological spaces if we endow $M,N$ with $\fa$-adic topology.
\end{prop}

\begin{proof}
Let $x\in M$ and $y:=\varphi(x)\in N$.
Let $U$ be any open neighborhood of $y$ in $N$.
Then there exists some $n$ such that $y+\fa^nN\subset U$.
Take $V=x+\fa^nM$ then it is an open neighborhood of $x$ in $M$,
and we have $\varphi(V)=y+\varphi(\fa^nM)=y+\fa^n\varphi(M)\subset y+\fa^nN\subset U$.
Therefore $\varphi$ is continuous.
\end{proof}

\begin{prop}
\label{Iwasawa-alg-isom-finite-level}
\uses{Iwasawa-alg-defn,linear-map-is-adic-continuous}
For each $n\geq 0$, there is an isomorphism of $\BZ_p$-algebras
$$
\BZ_p[\Gamma_n]\xrightarrow\sim\BZ_p[T]/\big((1+T)^{p^n}-1\big),\qquad
\gamma\mapsto 1+T.
$$
They are all free $\BZ_p$-modules of finite rank,
we endow them with $p$-adic topology.
By Proposition \ref{linear-map-is-adic-continuous}
we know that it is a homeomorphism of topological spaces.
\end{prop}

\begin{proof}
We have $\Gamma/\Gamma^{p^n}\cong\BZ/p^n\BZ$ as an abelian group,
and the image of $\gamma\in\Gamma$
in it is a generator of it. By abuse of notation we still denote the image of
$\gamma\in\Gamma$ in it by $\gamma$.
Then $\BZ[\Gamma/\Gamma^{p^n}]$ as a $\BZ$-module is free of rank $p^n$
and $\{\gamma^k\}_{0\leq k\leq p^n-1}$ is a basis of it.
Now it's easy to see that as a $\BZ_p$-module homomorphism,
$\BZ_p[\Gamma/\Gamma^{p^n}]\xrightarrow\sim\BZ_p[T]/\big((1+T)^{p^n}-1\big)$,
$\gamma^k\mapsto(1+T)^k$ is well-defined and is a $\BZ_p$-module isomorphism,
and preserves multiplication.
Therefore it is an isomorphism of $\BZ_p$-algebras.
\end{proof}

\begin{prop}
\label{Iwasawa-alg-isom}
\uses{Iwasawa-alg-isom-finite-level,isom-given-by-weierstrass-division}
There is an isomorphism of topological rings
$\BZ_p[[T]]\xrightarrow\sim\Lambda$
sending $1+T$ to $\gamma$,
where $\BZ_p[[T]]$ is endowed with $(p,T)$-adic topology.
\end{prop}

\begin{proof}
We prove that there is a natural isomorphism of $\BZ_p$-algebras
$$
\varprojlim_n\BZ_p[T]/\big((1+T)^{p^n}-1\big)
\xrightarrow\sim\BZ_p[[T]],
$$
and which is a homeomorphism of topological spaces,
where the topology of the left hand side is the subspace topology
of the product topology of the topology defined in Proposition
\ref{Iwasawa-alg-isom-finite-level},
and the topology of the right hand side it the $(p,T)$-adic topology.
From which we can obtain the isomorphism of topological rings
$\Lambda\xrightarrow\sim\BZ_p[[T]]$ given by $\gamma\mapsto 1+T$.

For simplicity of notation, denote $\varphi_n(T):=(1+T)^{p^n}-1$.
Clearly, for each $n\geq 0$ and for all $1\leq i\leq p^n-1$,
we have $\binom{p^n}{i}\in p\BZ$,
hence $\varphi_n(T)\in T^{p^n}+p\BZ[T]_{\deg\leq p^n-1}\subset(p,T)$,
in particular $\varphi_n(T)$ is a distinguished polynomial of degree $p^n$.
Moreover, $\varphi_{n+1}(T)=\varphi_n(T)\cdot\sum_{i=0}^{p-1}(1+T)^{p^ni}$
with $\sum_{i=0}^{p-1}(1+T)^{p^ni}\in(p,T)$,
hence $\varphi_n(T)\in(p,T)^{n+1}$ for all $n\geq 0$.

It's clear that $\BZ_p[T]_{\deg\leq p^n-1}\xrightarrow\sim\BZ_p[T]/(\varphi_n(T))$
is an isomorphism of $\BZ_p$-modules,
and the Weierstrass division (Proposition \ref{weierstrass-division}) implies that
$\BZ_p[T]_{\deg\leq p^n-1}\xrightarrow\sim\BZ_p[[T]]/(\varphi_n(T))$
is also an isomorphism of $\BZ_p$-modules,
therefore the natural ring homomorphism $\BZ_p[T]\hookrightarrow\BZ_p[[T]]$
induces an isomorphism of $\BZ_p$-algebras
$\BZ_p[T]/(\varphi_n(T))\xrightarrow\sim\BZ_p[[T]]/(\varphi_n(T))$
(Corollary \ref{isom-given-by-weierstrass-division}).

Since each $\BZ_p[[T]]/(\varphi_n(T))$ is a free $\BZ_p$-module of finite rank
endowed with the $p$-adic topology, we have
$\BZ_p[[T]]/(\varphi_n(T))\cong\varprojlim_m\BZ_p[[T]]/(p^m,\varphi_n(T))$,
where each $\BZ_p[[T]]/(p^m,\varphi_n(T))$ is endowed with discrete topology.
Therefore we have the following isomorphisms of rings as well as topological spaces:
\begin{align*}
\varprojlim_n\overbrace{\BZ_p[T]/(\varphi_n(T))}^{p\text{-adic topology}}
\xrightarrow\sim
\varprojlim_n\overbrace{\BZ_p[[T]]/(\varphi_n(T))}^{p\text{-adic topology}}
&\cong
\varprojlim_{m,n}\overbrace{\BZ_p[[T]]/(p^m,\varphi_n(T))}^{\text{discrete topology}} \\
&\stackrel{(*)}\cong
\varprojlim_k\overbrace{\BZ_p[[T]]/(p,T)^k}^{\text{discrete topology}}
\xrightarrow\sim
\overbrace{\BZ_p[[T]]}^{(p,T)\text{-adic topology}},
\end{align*}
here $(*)$ holds because $\{(p^m,\varphi_n(T))\}_{m\geq 1,n\geq 1}$
and $\{(p,T)^k\}_{k\geq 1}$ are \emph{cofinal}.
In fact, for each $k\geq 1$, and for any $m\geq k$ and $n\geq k$,
we have $(p^m,\varphi_n(T))\subset(p,T)^k$;
conversely, for each $m\geq 1$ and $n\geq 1$,
we have $(p,T)^{p^n}\subset(p,\varphi_n(T))$,
and $(p,\varphi_n(T))^m\subset(p^m,\varphi_n(T))$,
therefore for any $k\geq p^nm$, we have
$(p,T)^k\subset(p,T)^{p^nm}\subset(p,\varphi_n(T))^m\subset(p^m,\varphi_n(T))$.
\end{proof}

\subsection{Weierstrass preparation theorem}

This is WIP in \PR{21944}.

\begin{definition}
\label{distinguished-polynomial}
\leanok
If $(A,\fm,k)$ is a local ring, then a polynomial $f(X)=\sum_{i=0}^na_iX^i\in A[X]$
is called a \emph{distinguished polynomial}
if $a_n=1$ and $a_i\in\fm$ for all $0\leq i\leq n-1$.

(Mathlib: \docs{Polynomial.IsDistinguishedAt})
\end{definition}

\begin{prop}
\label{power-series-invertible-iff}
\leanok
Let $A$ be a ring, and let $f(X)=\sum_{n=0}^\infty a_nX^n\in A[[X]]$
be a formal power series.
Then $f(X)\in A[[X]]^\times$ if and only if $a_0\in A^\times$.

(Mathlib: \docs{PowerSeries.isUnit_iff_constantCoeff})
\end{prop}

\begin{proof}
\leanok
If $f(X)\in A[[X]]^\times$,
then there exists $g(X)=\sum_{n=0}^\infty b_nX^n\in A[[X]]$
such that $f(X)g(X)=1$, therefore by considering constant term we obtain
$a_0b_0=1$, hence $a_0\in A^\times$.
Conversely, if $a_0\in A^\times$, by multiplying $a_0^{-1}$ to $f(X)$
if necessary, we may assume that $a_0=1$
and $f(X)=1-Xf_1(X)$ for some $f_1(X)\in A[[X]]$.
Since $A[[X]]$ is $(X)$-adically complete and separated,
it's easy to see that $1+\sum_{k=0}^\infty X^kf_1(X)^k$
converges $(X)$-adically in $A[[X]]$ and which is the inverse of
$f(X)$.
\end{proof}

\begin{prop}[Weierstrass division]
\label{weierstrass-division}
\uses{power-series-invertible-iff}
\lean{PowerSeries.exists_isWeierstrassDivision,
PowerSeries.IsWeierstrassDivision.elim}
\leanok
Let $(A,\fm,k)$ be a complete local ring,
$g(X)=\sum_{i=0}^\infty a_iX^i\in A[[X]]\setminus\fm[[X]]$ be a formal power series
such that not all of its coefficients are in $\fm$.
Let $n\geq 0$ be the integer such that $a_n\in A\setminus\fm=A^\times$
and $a_i\in\fm$ for all $0\leq i\leq n-1$.
Then for any $f\in A[[X]]$,
there exists a unique formal power series
$q(X)\in A[[X]]$ and a unique polynomial $r(X)\in A[X]$ of degree $\leq n-1$
such that $f=gq+r$.
\end{prop}

\begin{proof}
\leanok
Write $g(X)=\sum_{i=0}^{n-1}a_iX^i+X^ng_1(X)$ for some $g_1(X)\in A[[X]]^\times$, and
$f(X)=\sum_{i=0}^{n-1}b_iX^i+X^nf_1(X)$ for some $f_1(X)\in A[[X]]$.
We construct a sequence $(q_k)_{k=1}^\infty$ inductively in $A[[X]]$,
such that $f-gq_k\in A[X]_{\deg\leq n-1}+\fm^k[[X]]$,
and such that $q_{k+1}(X)-q_k(X)\in\fm^k[[X]]$.
We construct $q_1(X):=f_1(X)g_1(X)^{-1}$.
Since $a_i\in\fm$ for all $i\leq n-1$, we have
\begin{align*}
f(X)-g(X)q_1(X)&=f(X)-\left(\sum_{i=0}^{n-1}a_iX^i\right)q_1(X)
-X^nf_1(X) \\
&=\sum_{i=0}^{n-1}b_iX^i-\left(\sum_{i=0}^{n-1}a_iX^i\right)q_1(X)
\in A[X]_{\deg\leq n-1}+\fm[[X]].
\end{align*}
Suppose $q_k(X)$ is constructed, then we may write
$f(X)-g(X)q_k(X)=\sum_{i=0}^{n-1}b_i^{(k)}X^i+X^ns_k(X)$
for some $s_k(X)\in\fm^k[[X]]$, and we construct
$q_{k+1}(X):=q_k(X)+s_k(X)g_1(X)^{-1}$. Then we have
\begin{align*}
f(X)-g(X)q_{k+1}(X)&=\sum_{i=0}^{n-1}b_i^{(k)}X^i+X^ns_k(X)
-\left(\sum_{i=0}^{n-1}a_iX^i\right)s_k(X)g_1(X)^{-1}
-X^ns_k(X) \\
&=\sum_{i=0}^{n-1}b_i^{(k)}X^i-\left(\sum_{i=0}^{n-1}a_iX^i\right)s_k(X)g_1(X)^{-1}
\in A[X]_{\deg\leq n-1}+\fm^{k+1}[[X]].
\end{align*}

Since $A[[X]]$ is complete and separated according to the sequence
$\{\fm^k[[X]]\}_{k\geq 1}$ of ideals, there exists a unique limit $q(X)\in A[[X]]$
of the sequence $(q_k)_{k=1}^\infty$, which satisfies $r:=f-gq\in A[X]_{\deg\leq n-1}$.

To prove the uniqueness, suppose $q(X)\in A[[X]]$ and $r(X)\in A[X]_{\deg\leq n-1}$
such that $gq=r$, then we prove by induction that for any $k\geq 0$
we have $q,r\in\fm^k[[X]]$, which implies that $q=r=0$.
When $k=0$ there is nothing to prove. Suppose $q,r\in\fm^k[[X]]$ for some $k\geq 0$.
Then we have $(\sum_{i=0}^{n-1}a_iX^i)q(X)+X^ng_1(X)q(X)=r(X)$,
since $a_i\in\fm$ for all $i\leq n-1$, we obtain $r(X)\in\fm^{k+1}[X]_{\deg\leq n-1}$.
Multiply $g_1(X)^{-1}$ to both side,
we obtain $(\sum_{i=0}^{n-1}a_iX^i)q(X)g_1(X)^{-1}+X^nq(X)=r(X)g_1(X)^{-1}$,
therefore $q(X)\in\fm^{k+1}[[X]]$.
\end{proof}

\begin{cor}
\label{isom-given-by-weierstrass-division}
\uses{weierstrass-division}
\lean{Polynomial.IsDistinguishedAt.algEquivQuotient}
\leanok
Let $(A,\fm,k)$ be a complete local ring,
$g(X)=\sum_{i=0}^n a_iX^i\in A[X]$ be a polynomial
such that $a_n\in A\setminus\fm$ and $a_i\in\fm$ for all $i<n$.
Then the natural map $A[X]/(g)\to A[[X]]/(g)$
is an isomorphism.
\end{cor}

\begin{proof}
\leanok
Let $f\in A[[X]]$.
Then by Proposition \ref{weierstrass-division},
we may find a unique formal power series
$q(X)\in A[[X]]$ and a unique polynomial $r(X)\in A[X]$ of degree $\leq n-1$
such that $f=gq+r$.
Then $r$ is the unique inverse of $f$ under the natural map $A[X]/(g)\to A[[X]]/(g)$.
\end{proof}

\begin{prop}[Weierstrass preparation theorem]
\label{weierstrass-preparation}
\uses{weierstrass-division,distinguished-polynomial}
\lean{PowerSeries.exists_isWeierstrassFactorization,
PowerSeries.IsWeierstrassFactorization.elim}
\leanok
Let $(A,\fm,k)$ be a complete local ring.
Let $g(X)\in A[[X]]\setminus\fm[[X]]$ be a formal power series
such that not all of its coefficients are in $\fm$.
Then there is a unique distinguished polynomial $f(X)\in A[X]$
and a unique invertible formal power series $h(X)\in A[[X]]^\times$ such that $g=fh$.
\end{prop}

\begin{proof}
\leanok
Take $f(X)=X^n$ in Proposition \ref{weierstrass-division}, we obtain $q(X)\in A[[X]]$
and $r(X)\in A[X]_{\deg\leq n-1}$ such that $X^n=g(X)q(X)+r(X)$.
Since $g(X)=\sum_{i=1}^{n-1}a_iX^i+X^ng_1(X)$
with $a_i\in\fm$ for all $i\leq n-1$ and $g_1(X)\in A[[X]]^\times$,
we have $r(X)\in\fm[X]_{\deg\leq n-1}$, and by the construction in (ii)
we have $q(X)\in g_1(X)^{-1}+\fm[[X]]\subset A[[X]]^\times$.
Therefore take $h(X):=q(X)^{-1}\in A[[X]]^\times$
and $f(X):=X^n-r(X)$, then $f(X)$ is a distinguished polynomial of degree $n$,
and $g(X)=f(X)h(X)$ holds.

To prove the uniqueness, suppose $f(X)$ and $f'(X)$ are two distinguished polynomials
and $u(X)\in A[[X]]^\times$ such that $f'(X)=f(X)u(X)$.
Then $\overline u(X)\in k[[X]]^\times$ and we have
$\overline f{}'(X)=X^{\deg(f')}=\overline f(X)\overline u(X)
=X^{\deg(f)}\overline u(X)\in k[[X]]^\times$,
which forces that $\deg(f)=\deg(f')$ and $\overline u(X)=1$.
Therefore $f'(X)-f(X)\in A[X]_{\deg\leq\deg(f)-1}$
and $f'(X)=f(X)+(f'(X)-f(X))$ is a Weierstrass division of $f'$ by $f$,
on the other hand, $f'(X)=f(X)u(X)$ is also a Weierstrass division of $f'$ by $f$,
hence by the uniqueness of Weierstrass division
we have $u(X)=1$ and $f'(X)=f(X)$.

(\emph{Another proof.}
It is possible to prove Weierstrass preparation theorem using a form
of Hensel's lemma presented in \url{https://ncatlab.org/nlab/show/Hensel's+lemma}.)
\end{proof}

\subsection{Characteristic ideal}

\begin{prop}
\label{Lambda-is-reg-of-dim-2}
\uses{Iwasawa-alg-isom}
$\Lambda\cong\BZ_p[[T]]$ is a
Noetherian regular local ring of Krull dimension $2$.
\end{prop}

\begin{cor}
\label{Lambda-is-UFD}
\uses{Lambda-is-reg-of-dim-2,regular-local-ring-is-UFD}
Hence by Proposition \ref{regular-local-ring-is-UFD}, $\Lambda$ is a UFD
(or maybe one can check directly that the $I$-adic completion of a UFD is a UFD).
\end{cor}

\begin{cor}
\label{Lambda-ht-1-principal}
\uses{Lambda-is-UFD,UFD-iff-ht-1-principal,weierstrass-preparation}
Hence by Proposition \ref{UFD-iff-ht-1-principal}, any height $1$ prime $\fp$ of $\Lambda$
is principal.

{\rm(i)}
If the generator of $\fp$ is in $p\BZ_p[[T]]$, then
we must have $\fp=(p)$.

{\rm(ii)}
If the generator of $\fp$ is not in $p\BZ_p[[T]]$, then
by Proposition \ref{weierstrass-preparation} such $\fp$
has a unique generator which is a distinguished polynomial.
\end{cor}

Therefore, we have

\begin{prop}
\label{Lambda-module-structure}
\uses{Lambda-ht-1-principal,structure-thm}
If $X$ is a finitely generated torsion $\Lambda$-module,
then there exists a pseudo-isomorphism
$$
X\to\bigoplus_{i=1}^m\Lambda/(f_i^{b_i})\oplus\bigoplus_{j=1}^s\Lambda/(p^{n_j})
$$
where $f_1,\cdots,f_m$ are distinguished polynomials.
The characteristic ideal $\chaR_\Lambda(X)$ is generated by
$p^{\sum_{j=1}^sn_j}\prod_{i=1}^mf_i^{b_i}$
which is contained in $p^{\sum_{j=1}^sn_j}\BZ_p[[T]]$
but not in $p^{1+\sum_{j=1}^sn_j}\BZ_p[[T]]$.
\end{prop}

\begin{definition}
\label{iwasawa-mod-invariants}
\uses{Lambda-module-structure}
{\rm(i)}
The $\mu$-invariant of $X$
is defined to be $\mu(X):=\sum_{j=1}^sn_j$,
and the $\lambda$-invariant of $X$
is defined to be $\lambda(X):=\sum_{i=1}^mb_i\deg f_i$.

{\rm(ii)}
If $f\in\BZ_p[[T]]$ is not zero, then define
$\mu(f)$ be the integer such that
$f\in p^{\mu(f)}\BZ_p[[T]]$ but
$f\notin p^{1+\mu(f)}\BZ_p[[T]]$,
define $\lambda(f)$ be the leading degree of
$(p^{-\mu(f)}f\bmod p)\in\BF_p[[T]]$.
\end{definition}

Clearly, $\mu(X)$ and $\lambda(X)$ are equal to
$\mu(f)$ and $\lambda(f)$ if $\chaR_\Lambda(X)=(f)$.

\begin{prop}
\label{iwasawa-mod-invariants-2}
\uses{iwasawa-mod-invariants}
We have $\mu(X)=\sum_{i=0}^\infty\rank_{\BF_p[[T]]}X[p^{i+1}]/X[p^i]$,
and $\lambda(X)=\rank_{\BZ_p}X/X[p^\infty]$.
%and $\chaR_\Lambda(X)=(p^{\mu(X)}h_X(T))\subset\Lambda$, where $h_X(T)$
%is the characteristic polynomial of $T$ acts on $X\otimes_{\BZ_p}\BQ_p$.
\end{prop}

......

\subsection{Growth of coinvariant part}

\begin{prop}
\label{coinvariant-growth}
\uses{iwasawa-mod-invariants-2}
If $X$ is a finitely generated torsion
$\Lambda$-module such that $X/(\gamma^{p^n}-1)X$ is finite
for any $n\geq 0$, then
there exists some constant $\nu=\nu(X)$
such that for all sufficiently large $n$,
$\ord_p(\#(X/(\gamma^{p^n}-1)X))=\mu(X)p^n
+\lambda(X)n+\nu(X)$.
\end{prop}

\section{Arithmetic of $\BZ_p$-extensions}

\subsection{The class group of $\BZ_p$-extension of a number field}

Let $K$ be a number field, $p$ be a prime,
$K_\infty/K$ be a $\BZ_p$-extension.

\begin{definition}
\label{Zp-ext-Ln}
\uses{Zp-ext-def}
{\rm(i)}
For each $n\geq 0$ let $L_n$ be the $p^\infty$-Hilbert class field of $K_n$.
That is, the maximal unramified abelian extension of $K_n$ of exponent $p^\infty$.

{\rm(ii)}
Let $X_n:=\Gal(L_n/K_n)\cong\Cl(K_n)(p)$,
the maximal quotient of the class group $\Cl(K_n)$ which is $p^\infty$-torsion.
\end{definition}

\begin{definition}
\label{Zp-ext-L-inf}
\uses{Zp-ext-Ln}
{\rm(i)}
Let $L_\infty:=\bigcup_{n\geq 0}L_n=\bigcup_{n\geq 0}L_nK_\infty$,
then it is an unramified abelian pro-$p$ extension of $K_\infty$,
because each $L_nK_\infty/K_\infty$ is finite unramified abelian $p$-extension.

{\rm(ii)}
Let $L_\infty'$ be the maximal unramified abelian pro-$p$ extension of $K_\infty$,
that is, the compositum of all finite unramified abelian $p$-extensions of $K_\infty$.
\end{definition}

\begin{prop}
\label{Zp-ext-L-eq-L}
\uses{Zp-ext-L-inf}
$L_\infty'=L_\infty$.
\end{prop}

Note that $L_\infty'=L_\infty$ is a Galois extension of $K$.

\begin{proof}
It is easy to see that ``$\supset$'' holds.
As for ``$\subset$'', suppose $E$ is a finite
unramified abelian $p$-extension of $K_\infty$,
we want to prove $E\subset L_\infty$. The proof consists of the following steps:

(1) There exists an integer $n_0\geq 0$ such that $E/K_{n_0}$ is Galois.

(2) There exists an integer $n_1\geq n_0$ such that $\Gal(E/K_{n_1})$ is abelian.
So $\Gal(E/K_{n_1})\cong\Gal(K_\infty/K_{n_1})\times G$,
where $G$ is a finite abelian group, corresponding to some $E_{n_1}/K_{n_1}$
finite abelian extension, so that $E=K_\infty E_{n_1}$.

(3) There exists an integer $n_2\geq n_1$ such that $E_{n_1}K_{n_2}/K_{n_2}$
is a finite unramified abelian $p$-extension. Therefore $E_{n_1}K_{n_2}\subset L_{n_2}$,
hence $E\subset L_{n_2}K_\infty\subset L_\infty$.
\end{proof}

\begin{lem}\label{p:rank-growth}
Suppose $X$ has rank $r$ as a $\Lambda$-module,
then we have
$$
\rank_{\BZ_p}\left(X/((1+T^{p^n})-1)X\right)=rp^n+O(1)
$$
as $n\to\infty$. This is left as an exercise.
For example, if $X=\Lambda$, then $r=1$,
and $\rank_{\BZ_p}\big(X/((1+T^{p^n})-1)X\big)=p^n$.
\end{lem}

\begin{prop}\label{p:unr-outside-p}
If $K$ is any number field, $K_\infty/K$ is any $\BZ_p$-extension,
and $\fl$ is a prime of $K$ not lying over $p$. Then $\fl$ is unramified in $K_\infty/K$.
\end{prop}

\begin{proof}
Let $D_\fl$ be the decomposition subgroup of $\fl$ in $\Gamma:=\Gal(K_\infty/K)$.
Let $l=\chaR(\CO_K/\fl)\neq p$, then $K_\fl/\BQ_l$ is a finite extension,
and let $(K_\infty)_\fl:=\varinjlim(K_n)_\fl$, then
$D_\fl=\Gal((K_\infty)_\fl/K_\fl)$ is a subgroup of $\BZ_p$.

We have $K_\fl\subset K_\fl^{\mathrm{unr}}\subset K_\fl^{\mathrm{ab}}$,
and
$$
\Gal(K_\fl^{\mathrm{unr}}/K_\fl)\cong\widehat\BZ
=\prod_{q\text{ prime}}\BZ_q.
$$
So $K_\fl$ has at least one $\BZ_p$-extension, i.e. the unique unramified
$\BZ_p$-extension. If there are other $\BZ_p$-extensions of $K_\fl$,
then there exists a Galois extension of $K_\fl$ with Galois group
isomorphic to $\BZ_p^2$.

However, $\Gal(K_\fl^{\mathrm{ab}}/K_\fl^{\mathrm{unr}})$
doesn't have a quotient isomorphic to $\BZ_p$, because by
local class field theory,
$\Gal(K_\fl^{\mathrm{ab}}/K_\fl^{\mathrm{unr}})\cong\CO_{K_\fl}^\times\cong
(\text{a finite group})\times\BZ_l^{[K_\fl:\BQ_l]}$ which is a finite group
times a pro-$l$ group, obviously it doesn't have a quotient isomorphic to $\BZ_p$.
So there is only one $\BZ_p$-extension of $K_\fl$,
note that $(K_\infty)_\fl/K_\fl$ is either trivial or a $\BZ_p$-extension,
in both cases it must be contained in $K_\fl^{\mathrm{unr}}$.
\end{proof}

\begin{thm}[Iwasawa]\label{p:f-g-torsion2}
\uses{Zp-ext-L-eq-L,p:nak,p:rank-growth,p:unr-outside-p}
Suppose $K$ is any number field, and $K_\infty/K$ is any $\BZ_p$-extension.
Let $L_\infty$ be the maximal unramified abelian pro-$p$ extension of $K_\infty$,
let $X_\infty:=\Gal(L_\infty/K_\infty)$ which is a $\Lambda$-module,
where $\Lambda:=\BZ_p[[\Gamma]]$, isomorphic to $\BZ_p[[T]]$
by choosing a topological generator $\gamma$ of $\Gamma$,
and maps $T$ to $\gamma-1$. Then $X_\infty$ is a finitely generated torsion
$\Lambda$-module.
\end{thm}

\begin{proof}
Consider $\Gal(K_\infty/K_n)=\Gamma^{p^n}$ with topological generator
$\gamma^{p^n}$. Let $E_n$ be the maximal abelian extension of $K_n$
contained in $L_\infty$, so that $\Gal(L_\infty/E_n)=\Gal(L_\infty/K_n)'
=(\gamma^{p^n}-1)X_\infty$. Note that $X_\infty/TX_\infty\cong\Gal(E_0/K_\infty)$
is a finitely generated $\BZ_p$-module, so by Nakayama lemma
(Lemma \ref{p:nak}) we know that $X_\infty$ is finitely generated $\Lambda$-module.

Recall that if $r=\rank_\Lambda X_\infty$, then
$\rank_{\BZ_p}\left(X_\infty/(\gamma^{p^n}-1)X_\infty\right)=rp^n+O(1)$
as $n\to\infty$ (Lemma \ref{p:rank-growth}). So in order to prove
$X_\infty$ is torsion (i.e. $r=0$), we only need to prove
$\rank_{\BZ_p}\left(X_\infty/(\gamma^{p^n}-1)X_\infty\right)$ is bounded.

Let $\fp_1,\cdots,\fp_t$ be the primes of $K$ which are ramified in $K_\infty/K$.
Then $t$ is finite (by Proposition \ref{p:unr-outside-p}),
and number of primes of $K_\infty$ lying over $\fp_1,\cdots,\fp_t$
is also finite, because for each $i$, the index $[\Gamma:D_{\fp_i}]$
is finite. So let $s_n$ be the number of primes of $K_n$ which are
ramified in $K_\infty/K_n$, then $s_n$ is bounded.

Consider $E_n/K_n$. Recall that $L_n$ is the $p$-Hilbert class field
of $K_n$. Then $L_n\subset L_\infty$, $K_\infty\subset L_\infty$,
so $L_nK_\infty\subset E_n$. Let $I_1,\cdots,I_{s_n}$ be the inertia
subgroups of $\Gal(E_n/K_n)$ for the ramified primes.
For $1\leq j\leq s_n$, $I_j\cap\Gal(E_n/K_\infty)=\{1\}$,
because $E_n/K_\infty$ is unramified. Therefore $I_j$ maps injectively
to a closed subgroup of $\Gamma^{p^n}$
via the map $\Gal(E_n/K_n)\to\Gal(K_\infty/K_n)$,
in particular, $I_j$ is isomorphic to $\BZ_p$.

Now we come to the key point. Let $I:=I_1\cdots I_{s_n}\subset\Gal(E_n/K_n)$,
then $\rank_{\BZ_p}I\leq s_n$, and $E_n^I$ is the maximal
unramified abelian pro-$p$ extension of $K_n$, so $E_n^I=L_n$,
hence $\rank_{\BZ_p}\Gal(E_n/K_n)\leq s_n$,
because $\Gal(L_n/K_n)\cong\Cl(K_n)(p)$ is a finite group.
Therefore $\rank_{\BZ_p}\Gal(E_n/K_\infty)\leq s_n-1$,
because $\Gal(K_\infty/K_n)=\Gamma^{p^n}$ is of $\BZ_p$-rank $1$.
Note that $\Gal(E_n/K_\infty)\cong X_\infty/(\gamma^{p^n}-1)X_\infty$,
so $\rank_{\BZ_p}\left(X_\infty/(\gamma^{p^n}-1)X_\infty\right)$ is bounded.
\end{proof}

\subsection{Recover the finite level of class group from infinite level}

Back to $K_n$ and $K_\infty$.
Recall that $X_\infty:=\Gal(L_\infty/K_\infty)=\varprojlim\Gal(L_n/K_n)$,
and $X_n:=\Gal(L_n/K_n)\cong\Cl(K_n)(p)$.

Let $\Gamma:=\Gal(K_\infty/K)\cong\BZ_p$, choose a topological
generator $\gamma$ of $\Gamma$ (or equivalently, $\gamma\in\Gamma$
such that $\gamma|_{K_1}$ is nontrivial).

\begin{prop}
\label{X-inf-coinv-is-X-n}
\uses{p:p-prime}
(This is incorrect ...)
For each $n\geq 0$ there is an isomorphism
$X_\infty/(\gamma^{p^n}-1)X_\infty\xrightarrow\sim
X_n$.
\end{prop}

\begin{proof}
Let $G:=\Gal(L_\infty/K)$.
We claim that $(\gamma-1)X_\infty$ is a closed normal subgroup of $G$
and $X_\infty/(\gamma-1)X_\infty\cong X_0=\Cl(K)(p)$. We have a group extension
$$
0\to X_\infty\to G\to\Gamma\to 1,
$$
which induces
$$
0\to X_\infty/(\gamma-1)X_\infty\to G/(\gamma-1)X_\infty\to\Gamma\to 1,
$$
so $G/(\gamma-1)X_\infty$ is abelian, and $(\gamma-1)X_\infty$
is a closed normal subgroup.
The proof of Proposition \ref{p:p-prime} can be easily modified to
show that $(\gamma-1)X_\infty=G'$.
Let $E=L_\infty^{G'}$ be the maximal abelian extension
of $K$ contained in $L_\infty$. Let $I_\fP$ be the inertia
subgroup of $\Gal(E/K)$ for a prime $\fP\mid\fp$,
then $L_0=E^{I_\fP}$. Let $H=\Gal(E/K_\infty)$
(so that $K_\infty=E^H$), then $L_0\cap K_\infty=K$,
$H\cap I_\fP=\{1\}$, so $E=L_0K_\infty$,
and $\Gal(E/K_\infty)\xrightarrow\sim\Gal(L_0/K)$
is a natural isomorphism.
Hence we conclude that $X_\infty/(\gamma-1)X_\infty\xrightarrow\sim X_0$.

In general, consider $L_n/K_n$, the $p$-Hilbert class field of $K_n$,
so that $\Gal(L_n/K_n)=:X_n\cong\Cl(K_n)(p)$.
We have $\Gal(K_\infty/K_n)\cong\Gamma^{p^n}\cong p^n\BZ_p$
with topological generator $\gamma^{p^n}$.
We have $K_\infty/K_n$ is ramified at only one place and is
totally ramified. So similarly, we get $X_\infty/(\gamma^{p^n}-1)X_\infty\xrightarrow\sim
X_n$.
\end{proof}

We state a group theory result.
Suppose $\CG$ is any group, $\CX$ is a normal abelian subgroup of $\CG$.
Then $\CG/\CX$ acts on $\CX$ as follows:
if $\sigma\in{\CG/\CX}$, lift $\sigma$ to an element $\widetilde\sigma$ in $\CG$.
Then define for all $x\in\CX$,
$\sigma(x):=\widetilde\sigma x\widetilde\sigma^{-1}$.
Note that $\sigma(x)x^{-1}=\widetilde\sigma x\widetilde{\sigma}^{-1}x^{-1}\in\CG'$.

Assume $\CG/\CX$ is cyclic with a generator $g$,
then $\CG'=\{g(x)x^{-1}\mid x\in \CX\}$.
We write $\CX$ additively, then $g(x)x^{-1}$ becomes $g(x)-x=(g-1)x$.

If we view $g-1$ as an element in $\BZ[\CG/\CX]$,
and view $\CX$ as a $\BZ[\CG/\CX]$-module,
then $\{g(x)-x\mid x\in \CX\}=(g-1)\CX$ is
a $\BZ[\CG/\CX]$-submodule of $\CX$. Hence $(g-1)\CX$ is a normal subgroup of $\CG$.

\begin{prop}\label{p:p-prime}
$\CG'=(g-1)\CX$.
\end{prop}
\begin{proof}
``$\supset$'' is trivial. As for ``$\subset$'',
we have an exact sequence
\[
0\rightarrow \CX/{(g-1)\CX}\rightarrow\CG/{(g-1)\CX}\rightarrow\CG/\CX\rightarrow 0,
\]
and $\CG/(g-1)\CX$ is a central extension of $\CG/\CX$ (which is cyclic)
by $\CX/{(g-1)\CX}$, so it is abelian, so $\CG'\subset(g-1)\CX$.
\end{proof}

\appendix

\section{Known results in mathlib}

\subsection{Rings}

\begin{itemize}
\item
Commutative ring with unit \docs{CommRing}
\item
Field \docs{Field}
\begin{itemize}
\item
assertion that a ring is a field \docs{IsField}
\end{itemize}
\item
assertion that a ring is an integral domain \docs{IsDomain}
\item
assertion that a ring is PID: \docs{IsDomain} + \docs{IsPrincipalIdealRing}
\item
assertion that a ring is UFD: \docs{IsDomain} + \docs{UniqueFactorizationMonoid}
\item
Noetherian ring \docs{IsNoetherianRing}
\begin{itemize}
\item
finitely many minimal prime ideals
\docs{minimalPrimes.finite_of_isNoetherianRing}
\item
finitely many minimal prime over-ideals
\docs{Ideal.finite_minimalPrimes_of_isNoetherianRing}
\end{itemize}
\item
Artin ring \docs{IsArtinianRing}
\begin{itemize}
\item
it is also Noetherian \docs{instIsNoetherianRingOfIsArtinianRing}
(instance, shouldn't need to call directly)
\end{itemize}
\item
Characteristic of a ring \docs{ringChar}, exponential characteristic \docs{ringExpChar}
\begin{itemize}
\item
assertion that a ring is of specific characteristic
\docs{CharZero}, \docs{CharP}, \docs{ExpChar}
\end{itemize}
\item
Krull dimension of a ring \docs{ringKrullDim}
\begin{itemize}
\item
assertion that a ring is of Krull dimension $\leq n$
\docs{Ring.KrullDimLE}
\item
assertion that a ring is of Krull dimension $\leq 1$
\docs{Ring.DimensionLEOne}
\end{itemize}
\end{itemize}

\subsection{Ideals}

\begin{itemize}
\item
Ideal of a ring \docs{Ideal}
\begin{itemize}
\item
assertion that an ideal is principal \docs{Submodule.IsPrincipal}
\item
assertion that an ideal is a prime ideal \docs{Ideal.IsPrime}
\begin{itemize}
\item
\docs{Ideal.Quotient.isDomain_iff_prime}
\end{itemize}
\item
assertion that an ideal is a maximal ideal \docs{Ideal.IsMaximal}
\begin{itemize}
\item
\docs{Ideal.Quotient.maximal_ideal_iff_isField_quotient}
\item
\docs{Ideal.Quotient.field} (instance, shouldn't need to call directly)
\end{itemize}
\end{itemize}
\item
Height of an ideal \docs{Ideal.height},
height of a prime ideal \docs{Ideal.primeHeight}
\begin{itemize}
\item
assertion that an ideal is the whole ring or of finite height \docs{Ideal.FiniteHeight}
\end{itemize}
\item
the \verb|Type| of prime ideals of a ring $\Spec(R)$ \docs{PrimeSpectrum}
\item
set of minimal primes \docs{minimalPrimes},
set of minimal prime over-ideals \docs{Ideal.minimalPrimes}
\end{itemize}

\subsection{Modules}

\begin{itemize}
\item
Support of a module $\Supp(M)$ \docs{Module.support}
\item
Annihilator of a module $\Ann(M)$ \docs{Module.annihilator}
\item
$M^*=\Hom_\BZ(M,\BQ/\BZ)$ \docs{CharacterModule}
\item
Associated primes of a module $\Ass(M)$ \docs{associatedPrimes}
\item
Finitely generated module \docs{Module.Finite}
\item
Free module \docs{Module.Free}
\item
Projective module \docs{Module.Projective}
\item
Injective module \docs{Module.Injective}
\item
Flat module \docs{Module.Flat}
\item
Torsion module \docs{Module.IsTorsion}
\item
Torsion submodule \docs{Submodule.torsion}
\begin{itemize}
\item
Torsion-free module \docs{Submodule.torsion}\verb| R M = ⊥|
\end{itemize}
\item
Noetherian module \docs{IsNoetherian}
\item
Artin module \docs{IsArtinian}
\item
assertion that a module is of finite length \docs{IsFiniteLength}
\begin{itemize}
\item
in the statement of theorems use \docs{IsNoetherian} + \docs{IsArtinian} instead
\item
if and only if exists composition series
\docs{isFiniteLength_iff_exists_compositionSeries}
\item
length of a module $\ell_A(M)$ \docs{Module.length}
\end{itemize}
\item
composition series \docs{CompositionSeries}
\begin{itemize}
\item
usage:

\verb|∃ (s : |\docs{CompositionSeries}\verb| (|\docs{Submodule}%
\verb| R M)), |\docs{RelSeries.head}\verb| s = ⊥ ∧ |\docs{RelSeries.last}\verb| s = ⊤|
\end{itemize}
\end{itemize}

\subsection{Number theory}

\begin{itemize}
\item
assertion that a field is a number field (i.e.~finite extension of $\BQ$)
\docs{NumberField}
\item
ring of integers $\CO_K$ \docs{NumberField.RingOfIntegers}
\item
assertion that a ring is a Dedekind domain \docs{IsDedekindDomain}
\begin{itemize}
\item
unique factorization of ideals \docs{Ideal.uniqueFactorizationMonoid}
(instance, shouldn't need to call directly)
\end{itemize}
\item
fraction ideals \docs{FractionalIdeal}
\begin{itemize}
\item
usage: if $R$ is a domain, $K$ is fraction field of $R$,
then the \verb|Type| of fraction ideals in $K$ is:

\docs{FractionalIdeal}\verb| (|\docs{nonZeroDivisors}\verb| R) K|
\item
$\ord_\fp(\fa)$ \docs{FractionalIdeal.count}
\end{itemize}
\item
ideal class group \docs{ClassGroup}
\begin{itemize}
\item
finite \docs{NumberField.RingOfIntegers.instFintypeClassGroup}
(instance, shouldn't need to call directly)
\item
class number \docs{NumberField.classNumber}
\end{itemize}
\item
places of a number field $K$:
\docs{AbsoluteValue}\verb| K ℝ|, \docs{NumberField.place}
\begin{itemize}
\item
infinite places \docs{NumberField.InfinitePlace}
\item
real and complex places
\docs{NumberField.InfinitePlace.IsReal},
\docs{NumberField.InfinitePlace.IsComplex}
\item
$r_1,r_2$
\docs{NumberField.InfinitePlace.nrRealPlaces},
\docs{NumberField.InfinitePlace.nrComplexPlaces}
\item
$r_1+r_2-1$
\docs{NumberField.Units.rank}
\item
Dirichlet unit theorem
\docs{NumberField.Units.rank_modTorsion}
\end{itemize}
\item
assertion that a field extension is abelian
\PR{23669}
\item
assertion that a field extension is cyclotomic
\docs{IsCyclotomicExtension}
\begin{itemize}
\item
cyclotomic field $K(\mu_n)$
\docs{CyclotomicField}
\end{itemize}
\item
$p$-adic cyclotomic character:
if $\mu_{p^\infty}\subset L$
then $\chi_\cyc:\Aut(L)\to\BZ_p^\times$
\PR{21934}
\end{itemize}

\begin{thebibliography}{XXXX}
\bibitem[Nek06]{Mek06}
J. Nekov\'{a}\v{r}.
\emph{Selmer complexes}.
Ast\'erisque No. 310 (2006), viii+559 pp.
\bibitem[NSW08]{NSW08}
J. Neukirch, A. Schmidt, K. Wingberg.
\emph{Cohomology of number fields}.
Second edition.
Grundlehren Math. Wiss., 323.
Springer-Verlag, Berlin, 2008. xvi+825 pp.
\bibitem[Was97]{Was97}
L. C. Washington.
\emph{Introduction to cyclotomic fields}.
Second edition.
Grad. Texts in Math., 83.
Springer-Verlag, New York, 1997. xiv+487 pp.
\end{thebibliography}
