% In this file you should put the actual content of the blueprint.
% It will be used both by the web and the print version.
% It should *not* include the \begin{document}
%
% If you want to split the blueprint content into several files then
% the current file can be a simple sequence of \input. Otherwise It
% can start with a \section or \chapter for instance.

\tableofcontents

\section*{Known results in mathlib}

\subsection{Property of rings}

\begin{itemize}
\item
Commutative ring with unit \docs{CommRing}
\item
Field \docs{Field}
\begin{itemize}
\item
assertion that a ring is a field \docs{IsField}
\end{itemize}
\item
assertion that a ring is an integral domain \docs{IsDomain}
\item
assertion that a ring is PID: \docs{IsDomain} + \docs{IsPrincipalIdealRing}
\item
Noetherian ring \docs{IsNoetherianRing},
Noetherian module \docs{IsNoetherian}
\begin{itemize}
\item
finitely many minimal prime ideals
\docs{minimalPrimes.finite_of_isNoetherianRing}
\item
finitely many minimal prime over-ideals
\docs{Ideal.finite_minimalPrimes_of_isNoetherianRing}
\end{itemize}
\item
Artin ring \docs{IsArtinianRing},
Artin module \docs{IsArtinian}
\item
Characteristic of a ring \docs{ringChar}, exponential characteristic \docs{ringExpChar}
\begin{itemize}
\item
assertion that a ring is of specific characteristic
\docs{CharZero}, \docs{CharP}, \docs{ExpChar}
\end{itemize}
\item
Krull dimension of a ring \docs{ringKrullDim}
\begin{itemize}
\item
assertion that a ring is of Krull dimension $\leq n$
\docs{Ring.KrullDimLE}
\end{itemize}
\end{itemize}

\subsection{Property of ideals}

\begin{itemize}
\item
Ideal of a ring \docs{Ideal}
\begin{itemize}
\item
assertion that an ideal is principal \docs{Submodule.IsPrincipal}
\item
assertion that an ideal is a prime ideal \docs{Ideal.IsPrime}
\begin{itemize}
\item
\docs{Ideal.Quotient.isDomain_iff_prime}
\end{itemize}
\item
assertion that an ideal is a maximal ideal \docs{Ideal.IsMaximal}
\begin{itemize}
\item
\docs{Ideal.Quotient.maximal_ideal_iff_isField_quotient}
\item
\docs{Ideal.Quotient.field} (instance, shouldn't need to call directly)
\end{itemize}
\end{itemize}
\item
Height of an ideal \docs{Ideal.height},
height of a prime ideal \docs{Ideal.primeHeight}
\begin{itemize}
\item
assertion that an ideal is the whole ring or of finite height \docs{Ideal.FiniteHeight}
\end{itemize}
\item
the \texttt{Type} of prime ideals of a ring $\Spec(R)$ \docs{PrimeSpectrum}
\item
set of minimal primes \docs{minimalPrimes},
set of minimal prime over-ideals \docs{Ideal.minimalPrimes}
\end{itemize}

\subsection{Property of modules}

\begin{itemize}
\item
Support of a module $\Supp(M)$ \docs{Module.support}
\item
Annihilator of a module $\Ann(M)$ \docs{Module.annihilator}
\item
$M^*=\Hom_\BZ(M,\BQ/\BZ)$ \docs{CharacterModule}
\item
Associated primes of a module $\Ass(M)$ \docs{associatedPrimes}
\item
assertion that a module is of finite length \docs{IsFiniteLength}
\begin{itemize}
\item
length of a module $\ell_A(M)$ \textbf{UNDEFINED !!!}
\end{itemize}
\end{itemize}

\section{Structure of Iwasawa module}

\subsection{Preliminaries on Noetherian rings}

\begin{prop}
\label{noeth ring filtration}
...
\end{prop}

\subsection{Characteristic ideal}

\begin{prop}
\label{char ideal}
\uses{noeth ring filtration}
Let $A$ be a Noetherian ring, $M$ be a
finitely generated torsion $A$-module.
Then for any height one prime $\fp$ of $A$,
$M_\fp$ is an $A_\fp$-module of finite length. Moreover, there are only finitely
many height one primes $\fp$ of $A$ such that $M_\fp\neq 0$.
\end{prop}

\begin{proof}
By Proposition \ref{noeth ring filtration},
we may let $0=M_0\subset M_1\subset\cdots\subset M_n=M$ be a filtration of $M$
such that for each $1\leq i\leq n$,
$M_i/M_{i-1}\cong A/\fp_i$ for some prime ideal $\fp_i$ of $A$.
Note that if $\fp,\fq$ are prime ideals of $A$,
then $(A/\fp)_\fq\neq 0$ if and only if $\fp\subset\fq$.
Therefore by $M$ is torsion $A$-module, we obtain that $\height(\fp_i)\geq 1$
for all $1\leq i\leq n$,
and if $\fp$ is a height one prime, then $M_\fp\neq 0$
if and only if $\fp_i\subset\fp$ for some $i$,
by height considerations this means that $\fp_i=\fp$ for some $i$,
hence such $\fp$ are only finitely many.

To prove $\ell_{A_\fp}(M_\fp)<\infty$,
we only need to show that if $\fp,\fq$ are prime ideals of $A$
such that $\height(\fp)\geq 1$ and $\height(\fq)=1$,
then $(A/\fp)_\fq$ is an $A_\fq$-module of finite length.
In fact, by height considerations we know that $(A/\fp)_\fq\neq 0$
if and only if $\fp=\fq$, in this case $(A/\fq)_\fq=A_\fq/\fq A_\fq
=k(\fq)$ is the residue field of $\fq$, which is
an $A_\fq$-module of length one.

(\emph{Another proof without using Proposition \ref{noeth ring filtration}.}
Note that $M_\fp=0$ for all minimal prime ideals of $A$,
therefore if $\fp$ is of height one such that $M_\fp\neq 0$,
then $\fp$ is a minimal element in $\Supp(M)$, hence $\fp\in\Ass(M)$
which is a finite set.
So there are only finitely
many height one primes $\fp$ of $A$ such that $M_\fp\neq 0$.

Suppose $\fp$ is a height one prime such that $M_\fp\neq 0$.
To prove that $M_\fp$ is an $A_\fp$-module of finite length,
we only need to prove that the ring $A_\fp/\Ann_{A_\fp}(M_\fp)$
is Artinian. Note that $\Ann_{A_\fp}(M_\fp)=\Ann_A(M)_\fp$,
hence $A_\fp/\Ann_{A_\fp}(M_\fp)=(A/\Ann_A(M))_\fp$
whose prime ideals are one-to-one correspondence to prime ideals
of $A$ between $\Ann_A(M)$ and $\fp$,
i.e.~prime ideals in $\Supp(M)$ which is contained in $\fp$.
Such ideal can only be $\fp$ itself,
since $M$ is torsion, every prime ideal in $\Supp(M)$ has height $\geq 1$.
Hence $A_\fp/\Ann_{A_\fp}(M_\fp)$
is Artinian.)
\end{proof}

In particular, this allows us to define the \emph{characteristic ideal} of $M$.

\begin{definition}
\label{def:char ideal}
\uses{char ideal}
Let $A$ be a Noetherian ring, $M$ be a
finitely generated torsion $A$-module.
The \emph{characteristic ideal} of $M$,
denoted by $\chaR_A(M)$, or simply $\chaR(M)$ if there is no risk of confusion, is defined to be
$$
\chaR_A(M):=\prod_{\substack{\fp\in\Spec(A)\\
\height(\fp)=1}}\fp^{\ell_{A_\fp}(M_\fp)}.
$$
\end{definition}

\begin{prop}
\label{char ideal additive}
\uses{def:char ideal}
Let $A$ be a Noetherian ring.
Let $0\to M'\to M\to M''\to 0$ be a short exact sequence of finitely generated $A$-modules. Then $M$ is $A$-torsion if and only if $M'$ and $M''$ are $A$-torsion.
If $M$ is $A$-torsion, then $\chaR_A(M)=\chaR_A(M')\chaR_A(M'')$.
\end{prop}

\begin{proof}
Since localization is exact, for any prime ideal $\fp$ of $A$,
the $0\to M_\fp'\to M_\fp\to M_\fp''\to 0$ is exact.
Let $\fp$ runs over all minimal prime ideals of $A$,
we obtain that $M$ is $A$-torsion if and only if $M'$ and $M''$ are $A$-torsion.
Also, we have $\ell_{A_\fp}(M_\fp)=\ell_{A_\fp}(M_\fp')+\ell_{A_\fp}(M_\fp'')$,
hence $\chaR_A(M)=\chaR_A(M')\chaR_A(M'')$ holds.
\end{proof}

\subsection{Pseudo-null module}

\begin{definition}
\label{def:pseudo null}
Let $A$ be a Noetherian ring.

{\rm(i)}
A finitely generated $A$-module $M$ is called a \emph{pseudo-null} $A$-module,
if $M_\fp=0$ for all prime ideals $\fp$ of $A$ of height $\leq 1$.

{\rm(ii)}
An $A$-linear homomorphism $f:M\to N$ between finitely generated $A$-modules
is called a \emph{pseudo-isomorphism} (\emph{pis} for short),
if its kernel and cokernel are pseudo-null $A$-modules.

{\rm(iii)}
Two finitely generated $A$-modules $M,N$ are called
\emph{pseudo-isomorphic} (\emph{pis} for short),
denoted by $M\sim_\pis N$ or simply $M\sim N$,
if there exists a pseudo-isomorphism from $M$ to $N$.
\end{definition}

\begin{remark}
We warn the reader that $M\sim N$ not necessarily implies $N\sim M$.
\end{remark}

\begin{prop}
\label{pseudo null criterion}
\uses{def:pseudo null}
Let $A$ be a Noetherian ring, $M$ be a finitely generated $A$-module.
If $A$ is of Krull dimension $\leq 1$, then $M$ is pseudo-null
if and only if $M=0$.
If $A$ is of Krull dimension $2$, then $M$ is pseudo-null
if and only if the cardinality of $M$ is finite (???).
\end{prop}

\subsection{Structure theorem}

\begin{definition}
\label{def:ht-1-localization-is-PID}
Let $A$ be a Noetherian ring.
We say that the height one localization of $A$ is PID (?), if
\begin{equation}
\label{e:ht-1-localization-is-PID}
\begin{array}{l}
\text{For any finitely many height one primes $\fp_1,\cdots,\fp_r$ of $A$,}\\
\text{let $S:=A\setminus\bigcup_{i=1}^r\fp_i$, then $S^{-1}A$ is a PID.}
\end{array}
\end{equation}
\end{definition}

\begin{lem}
\label{pis-iff}
\uses{char ideal}
Let $A$ be a Noetherian ring and let
$M,N$ be finitely generated torsion $A$-modules.
Let $\Sigma=\{\fq_1,\cdots,\fq_r\}
=\{\fq\in\Supp(M)\cup\Supp(N)\mid\height(\fq)=1\}$
(by Proposition \ref{char ideal} this is a finite set).
Let $S:=A\setminus\bigcup_{i=1}^r\fq_i$ which is a multiplicative subset of $A$.
Let $f:M\to N$ be an $A$-module homomorphism.
Then $f$ is a pseudo-isomorphism if and only if $S^{-1}f:S^{-1}M\to S^{-1}N$
is an isomorphism.
\end{lem}

\begin{proof}
Since the height one support of $\ker(f)$ and $\coker(f)$
are contained in $\Sigma$,
and since $S^{-1}\ker(f)=\ker(S^{-1}f)$,
$S^{-1}\coker(f)=\coker(S^{-1}f)$ (localization is exact),
we only need to prove that if $M$ is a finitely generated torsion $A$-module
whose height one support is contained in $\Sigma$,
then $S^{-1}M=0$ if and only if $M$ is pseudo-null (equivalently, $M_\fq=0$
for all $\fq\in\Sigma$):
``$\Rightarrow$'': Clear.
``$\Leftarrow$'': For all $\fq\in\Sigma$, $M_\fq=0$ means that $\Ann(M)\not\subset\fq$,
since $\fq$ are prime ideals, we have $\Ann(M)\not\subset\bigcup_{\fq\in\Sigma}\fq$,
so $S^{-1}M=0$.
\end{proof}

\begin{prop}[Structure theorem of finitely generated torsion $A$-modules]
\label{structure thm}
\uses{def:ht-1-localization-is-PID,char ideal}
Let $A$ be a Noetherian ring satisfying \eqref{e:ht-1-localization-is-PID}
and let $M$ be a finitely generated torsion $A$-module.
Then there exist height one primes $\fp_1,\cdots,\fp_s$ of $A$
and positive integers $k_1,\cdots,k_s$, such that there exists a pseudo-isomorphism
$M\to\bigoplus_{i=1}^s A/\fp_i^{k_i}$.
Moreover, the sequence $(\fp_i^{k_i})_{i=1}^s$ is unique up to ordering.
\end{prop}

\begin{proof}
Let
$\Sigma=\{\fq_1,\cdots,\fq_r\}=\{\fq\in\Supp(M)\mid\height(\fq)=1\}$
(by Proposition \ref{char ideal} this is a finite set),
and let $S=A\setminus\bigcup_{i=1}^r\fq_i$.
Then $S^{-1}M$ is a finitely generated $S^{-1}A$-module,
and is torsion (for example, since $\Hom_{S^{-1}A}(S^{-1}M,S^{-1}A)
=S^{-1}\Hom_A(M,A)=0$).

Note that the prime ideals $\fP$ of $S^{-1}A$ are one-to-one correspondence
to prime ideals $\fp$ of $A$ satisfying $\fp\cap S=\varnothing$
(i.e.~$\fp\subset\bigcup_{i=1}^r\fq_i$, i.e~$\fp\subset\fq_i$ for some $i$),
by $\fP=S^{-1}\fp$ and $\fp=\fP\cap A$.
In particular, $S^{-1}A$ is of dimension $\leq 1$.

By structure theorem of finitely generated torsion modules over a PID,
there exist primes $\fp_1,\cdots,\fp_s$ of $A$
satisfying $\fp_i\cap S=\varnothing$,
and positive integers $k_1,\cdots,k_s$, such that
there exists an isomorphism
$g:S^{-1}M\xrightarrow\sim\bigoplus_{i=1}^sS^{-1}(A/\fp_i^{k_i})$
of $S^{-1}A$-modules.
Since $S^{-1}M$ is torsion, the $\fp_i$'s must be of height one.
Since $\Hom_{S^{-1}A}(S^{-1}M,\bigoplus_{i=1}^sS^{-1}(A/\fp_i^{k_i}))
=S^{-1}\Hom_A(M,\bigoplus_{i=1}^sA/\fp_i^{k_i})$,
by multiplying an element of $S$ to $g$ if necessary
(this doesn't change the fact that $g$ is an isomorphism), we may find
an $A$-linear map $f:M\to\bigoplus_{i=1}^sA/\fp_i^{k_i}$
such that $g=S^{-1}f$.
Now by (i) we know that $f$ is a pseudo-isomorphism.

Conversely, if $(\fp_i^{k_i})_{i=1}^s$ is such that there exists a
pseudo-isomorphism
$M\to\bigoplus_{i=1}^s A/\fp_i^{k_i}$,
then enlarging $S$ if necessary, by Lemma \ref{pis-iff},
its localization
$S^{-1}M\to\bigoplus_{i=1}^sS^{-1}(A/\fp_i^{k_i})$
is an isomorphism of $S^{-1}A$-module, hence by
structure theorem of finitely generated torsion modules over a PID,
the $(\fp_i^{k_i})_{i=1}^s$ is unique up to ordering.
\end{proof}

\begin{prop}
\label{pis symm}
\uses{def:ht-1-localization-is-PID,pis-iff}
Let $A$ be a Noetherian ring satisfying \eqref{e:ht-1-localization-is-PID}.
Let $M,N$ be finitely generated torsion $A$-modules.
Then the followings are equivalent:
\begin{itemize}
\item[(a)]
There exists a pseudo-isomorphism $M\to N$.
\item[(b)]
For any height one prime $\fp$ of $A$, we have $M_\fp\cong N_\fp$.
\end{itemize}
In particular, if there exists a pseudo-isomorphism $M\to N$,
then there also exists a pseudo-isomorphism $N\to M$.
\end{prop}

\begin{proof}
(a)$\Rightarrow$(b): Clear.

(b)$\Rightarrow$(a): Let $\Sigma=\{\fq_1,\cdots,\fq_r\}
=\{\fq\in\Supp(M)\cup\Supp(N)\mid\height(\fq)=1\}$
(by Proposition \ref{char ideal} this is a finite set),
and let $S=A\setminus\bigcup_{i=1}^r\fq_i$.
Since $M_\fp\cong N_\fp$ for all height one primes $\fp$ of $A$,
the $S^{-1}M$ and $S^{-1}N$, being finitely generated torsion modules over
a PID $S^{-1}A$, are isomorphic. Say $g:S^{-1}M\xrightarrow\sim S^{-1}N$
is an isomorphism of $S^{-1}A$-modules.
Since $\Hom_{S^{-1}A}(S^{-1}M,S^{-1}N)=S^{-1}\Hom_A(M,N)$,
by multiplying an element of $S$ to $g$ if necessary
(this doesn't change the fact that $g$ is an isomorphism), we may find
an $A$-linear map $f:M\to N$
such that $g=S^{-1}f$.
Now by Lemma \ref{pis-iff} we know that $f$ is a pseudo-isomorphism.
\end{proof}

\subsection{Noetherian regular domain}

\begin{lem}
\label{semilocal PID}
Let $A$ be a semilocal (i.e.~only finitely many maximal ideals)
integral domain which is not a field,
such that for every maximal ideal $\fp$ of $A$, $A_\fp$ is a PID.
Then $A$ itself is a PID.
\end{lem}

\begin{proof}
...
\end{proof}

\begin{prop}
\label{regular domain is good}
\uses{semilocal PID}
A Noetherian regular domain satisfies \eqref{e:ht-1-localization-is-PID}.
\end{prop}

\begin{proof}
...
\end{proof}

\begin{remark}
In fact, a Noetherian integrally closed domain satisfies
\eqref{e:ht-1-localization-is-PID}.
\end{remark}
